\chapter*{Avant-propos}
\markboth{\normalfont\sffamily\textsc{Avant-propos}}{}
Le document comporte six (6) chapitres repartis en trois grande partie. La première partie ("Introduction") est pensée pour apporter les connaissances de base des différents domaines d'expertise impliqués dans cette étude ainsi que pour introduire de manière plus formelle le thème, ses spécificités et ses implications. Cette première partie contient les deux premiers chapitres portant sur les deux piliers de ce thème, respectivement le système de prêt et la science des données. L'objectif étant d'introduire les concepts fondamentaux qui englobent les différentes techniques que j'ai employé dans mes recherches.

La deuxième partie rentre dans les détails de cette méthodologie, cette partie se veut aussi détaillée que possible afin d'apporter une compréhension complète du travail éffectué. En partant des concepts généraux jusqu'aux utilisations et implémentations spécifiques que j'en ai fait j'espère démystifier le processus avant de présenter les résultats obtenu. Le chapitre 3 concerne exclusivement les manipulations et traitements appliqués aux données qui ont servi à dévélopper les différents modèles tandis que le chapitre 4 explique le fonctionnement et l'implémentation de ces modèles.
Enfin le chapitre 5 détaillera le processus d'évaluation des différents modèles entraînés avant de procéder à une évaluation comparatives des différents algorithmes.

Finalement la partie conclusion et le dernier chapitre visent à offir un résumé concis du document, à apporter des réponses à certaines questions que vous vous poserez probablement après votre lecture, à mettre en lumière les challenges rencontrés au cours des recherches et enfin à souligner des pistes de recherches potentielles pour de futures avancées concernant les systèmes de prédiction de défaut de paiement.

Je tenais à faire ce mémoire le plus explicite possible afin qu'aucune connaissance préalable ne soit nécéssaire pour comprendre les différents concepts. Toutefois, quelques connaissances de bases en statisques, theorie de la probabilité, algèbre linéaire et algorithmique à un niveau sophomore pourraient grandement faciliter la compréhension de certains concepts.

\clearpage