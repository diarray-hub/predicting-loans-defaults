\section{Introduction formelle du thème}
\label{chap1.section3}
Bien que ces nouvelles approches ne sont pas encore largement répendues et utilisées par les banques au Mali, elles sont beaucoup plus communes en occident et déjà adoptées par les grandes institutions. Donc des progrès déjà significatifs ont été réalisés dans le domaine, enfaîte la prediction des défauts de paiement est l'une des plus vielles application de l'apprentissage automatique, un classique.

Ce que je propose avec ce thème est donc une analyse de cet état de l'art et du flux de travail associé, pour celà j'ai implémenté certaines des dernières techniques de prétraitement et d'analyse des données ainsi que les modèles d'apprentissage automatiques qui ont atteint l'état de l'art dans cette tâche. J'ai voulu diverger de la majorité des chercheurs en testant aussi des algorithmes moins conventionnels et moins populaires pour la prediction de défaut de paiement, qui m'ont, au final, agréablement surpris par leur résultats au moment des tests.

Au total cinq (5) modèles d'apprentissage automatique différents ont été entraînés et testés, des valeurs sûres comme la régression logistique, la forêt aléatoire et le boost de gradient mais aussi les réseaux de neuronnes artificiels (souvent mis de côté pour ce genre de tâche en raison de leur complexité) et la détection d'anomalie qui à la base n'est pas un algorithme de classification mais qui à démontré la meilleure précision avec des données déséquilibrées. Le chapitre 4 va en profondeur dans les processus d'entraînement et les différences conceptuelles de chacun de ces modèles ainsi que les implémentations et stratégies que j'ai utilisé.

Ces cinq modèles ont tous été entraînés et testés suivant le même canevas ou \textit{pipeline} et avec le même ensemble de données, afin que les performances ne puisse différer que de part les caractéristiques intrinsèques de chaque modèle limitant leur performances après un certains nombre d'étapes d'optimisation ainsi évitant de biaser leur évaluation dans un système ou un modèle est avantagé par rapport aux autres.

L'ensemble de données en question est constituée d'exactement 100 variables indépendantes ou prédicateurs (\textbf{features}) et d'une variable dépendante (celle qu'on cherche à prédire, \textbf{target}), à savoir si oui ou non le client en question a remboursé son prêt. Totalisant 1.526.659 exemples étiquétés qui seront divisés en deux groupes pour l'entraînement et le test. Ces données ont été obtenu après une étape d'ingénierie de prédicateurs ou création de prédicateurs et une étape de sélection et de prétraitement des variables indépendantes (détaillées dans le chapitre 3) que j'ai appliqué aux données brutes fournies par Home Credit Group, une institution financière non bancaire internationale fondée en 1997 en République tchèque et dont le siège est aux Pays-Bas, lors d'une competition qu'ils organisaient sur la plateforme mondiale des scientifiques des données et d'ingénieurs en apprentissage automatique, \textbf{Kaggle} (\cite{herman2024home}). La société opère dans 9 pays et se concentre sur les prêts avec une attention particulière aux personnes ayant peu ou pas d'antécédents de crédit.

Ceci couvre en grosso modo ce qu'il faut savoir sur le travail éffectué. Dans le deuxième chapitre nous nous intéressons à la définition et à la compréhension des concepts phares de la science des données qui sont mentionnés ou utilisés dans le reste du document. Ce chapitre vise à familiariser le lecteur avec ces notions avant de rentrer dans la partie technique du document, vous pouvez le voir comme une sorte de glossaire mais un peu plus détaillé.