\section{Contexte}
\label{chap1.section1}
Dans notre société contemporaine, l'économie à l'échelle mondiale se répose de plus en plus sur les prêts. Cette situation est notamment favorisée par le fait qu'aujourd'hui toutes les économies se basent sur une forme de capitalisme\footnote{Le capitalisme est un système économique caractérisé par la propriété privée des moyens de production, des marchés concurrentiels et la recherche du profit.}, les économies socialistes\footnote{Le socialisme est un système politique et économique dans lequel la propriété et les moyens de production sont détenus en commun, généralement contrôlés par l'État ou le gouvernement. Le socialisme repose sur l’idée que la propriété commune ou publique des ressources et des moyens de production conduit à une société plus égalitaire.} étant souvent taxée comme restrictives de liberté et empêchant le dévéloppement économique des individus au profit de celui de l'état. Cette mentalité inspirée surtout par \textbf{\textit{"The american dream"}}, le rêve américain en français, a pas mal chambouler les normes, maintenant on a tous un peu l'idée que pour être quelqu'un il faut avoir une villa, une voiture et des bouts de terre à son nom.

Ce changement de mentalité a été le point de départ d'importants changements dans l'économie mondiale. Le rêve américain s'est démocratisé et est devenu un rêve mondial, apportant l'idée que chaque personne a la liberté et la possibilité de réussir et d’accéder à une vie meilleure. C'est donc à partir de là que le système de prêt est devenu réellement LE pilier de l'économie mondiale avec un impact absolument démésuré. D'après la banque mondiale, le crédit mondial au secteur privé représentait environ 97,6\% du PIB mondial en 2020 (\cite{worldbank2024domestic}).

Les prêts ont definitivement changé de statut et l'économie mondiale de visage, les individus emprûntent de l'argent pour s'acheter une maison, une boutique, une voiture, en bref vivre le rêve américain mais aussi pour faire des études supérieures. L'americain moyen passe typiquement les premières années de sa vie professionelle à rembourser son prêt étudiant. Au delà des individus les gouvernements contribuent à renforcer ce système souvent à travers des rélations complexes avec les banques.

Mais là où l'impact est le plus grand est sans aucun doute le secteur de l'entreprenariat privé. Il est vraiment impressionant de se dire qu'il y a des entreprises aujourd'hui évaluées à des centaines de millions voir des milliards de dollars qui partent vraiment juste d'une bonne idée et d'un investissement. Pendant la crise du coronavirus, les crédits ont été le bienfaiteur qui a sauvé des milliers de petites et moyennes entreprises à travers le monde et paradoxalement ils ont aussi été la raison qui a précipité la chutte de centaines d'entre elles.

Les dettes, c'est souvent le commencement de la ruine. Après la crise du covid-19, le FC Barcelone (incontestablement un des plus grands clubs de football de l'histoire) annoncait une dette globale de plus de 2 milliards d'euros, ce fut le début d'une période très difficile que le club traverse encore aujourd'hui. Pas seulement dans le domaine du sport, tellement d'entreprises et organisation ont dû déposer le bilan à cause de leur dettes, les prêts font les grands et les dettes les détruisent. On ne crée pas de richesse à partir de rien, ce qui fait que même les banques et les institutions prêteuse sont en réalité souvent très endettées. Pour rappel, la banque d'investissement des frères Lehman a déposé son bilan en raison de son exposition massive aux dettes hypothécaires à risque, ce qui a conduit à la grande crise financière de 2008.

C'est probablement après cet évènement que les gens et en particulier les américains ont réellement réalisé que l'énorme croissance économique que l'on a connu après la seconde guerre mondiale s'est faite sur des dettes cumulées énormes des grandes institutions et qu'on ferait mieux ne pas jouer avec ça. Suivant la crise plusieurs lois ont été votées ou modifiées un peu partout dans le monde afin d'eviter un tel désastre dans le futur, mettant l'accent sur la protection du client mais aussi la transparence des procedures.

Vous pensez peut-être que toutes ces choses sont des problèmes de la société contemporaine mais détrompez-vous, j'ai été surpris d'apprendre que bien qu'ils n'avaient jamais atteint un tel niveau d'importance auparavant différents systèmes de prêt ont en réalité soutenu les économies de plusieurs civilisations par le passé. Bien que ces concepts et lois ont évolué avec le temps, ils sont enfaîte extrêmement anciens. Autour de moins 2000 ans avant Jésus Christ en Mésopotamie, les temples et les palais prêtaient du grain aux agriculteurs et aux commerçants avec des formes de contrat incluant des taux d'interêt, des clauses et des délais, un peu comme les prêts d'aujourd'hui. Le \textbf{code d'Hammurabi}, un texte légal Babylonien composé entre 1755 et 1750 avant Jésus Christ incluait déjà des règles sur les prêts et les taux d'intérêt. Il s’agit du texte juridique le plus long, le mieux organisé et le mieux conservé de l’ancienne Mésopotamie. Il est écrit dans le vieux dialecte babylonien de l'akkadien, prétendument par Hammurabi, sixième roi de la première dynastie de Babylone.

Ça fait beaucoup de choses qu'un informaticien de la génération Z n'est pas vraiment censé savoir mais s'il y'a une chose que j'ai bien compris c'est qu'avec le model économique actuelle le système de prêt est à proteger absolument au risque de l'éffondrement de l'économie mondiale. Dans cette optique, une décision en particulier devient plus crucial que le reste: À qui accorder un prêt? Pour proteger à la fois le client et le prêteur, il est primordial de pouvoir déterminer avec précision quel client est ou sera en capacité de rembourser tel prêt. Voilà qui nous amène à notre thème maintenant, dans la prochaine section nous allons discuter le comment maintenant.