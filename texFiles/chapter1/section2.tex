\section{Approches}
\label{chap1.section2}
Les banques sont tenues de respecter les normes réglementaires qui imposent des pratiques de prêt prudentes. Pouvoir prédire de manière précise qui peut ou ne peut pas rembourser tel prêt aident à respecter ces normes et à éviter les pénalités.

Pour prédire les défauts de paiement, les banques analysaient traditionnellement  divers facteurs, parmi lesquels:

\begin{itemize}
    \item Historique de crédit: un enregistrement détaillé du comportement d’emprunt et de remboursement passé du demandeur.
    \item Revenu et emploi: un revenu constant et suffisant indique une capacité à rembourser le prêt.
    \item Ratio dette/revenu: il mesure le fardeau de la dette du demandeur par rapport à son revenu.
    \item Valeur de la garantie: pour les prêts garantis, la valeur de la garantie est cruciale en cas de défaut de paiement de l'emprunteur.
\end{itemize}

Typiquement une demande de prêt est analysée méticuleusement suivant ces différents aspects par des professionels du domaine qui en fonction de ces éléments decident d'accorder ou de refuser le prêt au demandeur. Vous l'aurez compris, les antécédents priment beaucoup dans la procedure, un jeune diplômé qui vient d'obtenir son premier contrat à durée indéterminée (CDI) n'a presque aucune chance de pouvoir obtenir un prêt, à moins d'avoir une garantie de fer.

Des méthodes un peu plus modernes peuvent également inclure des données comportementales, telles que les habitudes de dépenses et même l'activité sur les réseaux sociaux. Oui ça commence à faire un peu beaucoup, ce qui nous amène encore une fois directement au thème de ce mémoire. L'étude d'une demande de prêt est une procédure délicate qui augmente en complexité avec différents facteurs. Les données à analyser aussi croissent en complexité et surtout en quantité multipliant le risque d'errreur humaine dans la procédure.

Des mauvaise prévisions repétées peuvent avoir de graves répercussions:

\begin{itemize}
    \item Augmentation des pertes sur prêts: un nombre plus élevé de défauts de paiement a un impact direct sur les résultats financiers de la banque.
    \item Taux d’intérêt plus élevés : pour couvrir les pertes potentielles, les banques pourraient augmenter les taux d’intérêt pour tous les emprunteurs, rendant les prêts plus chers.
    \item Resserrement du crédit : les banques pourraient devenir plus conservatrices dans leurs prêts, ce qui rendrait plus difficile l’obtention de prêts pour les particuliers et les entreprises.
    \item Dommages à la réputation : des taux de défaut élevés peuvent nuire à la réputation d’une banque, affectant la confiance des clients et ses activités futures.
\end{itemize}

\textbf{L'apprentissage automatique} ou apprentissage machine, un sous-domaine de l'intelligence artificiel centré sur la création d'algorithme qui puisse extraire, interprêter et généraliser des connaissances à partir des données, a apporté de nouvelles méthodes visant à assister cette procédure par les machines, améliorant considérablement la précision et l’efficacité. Ces nouvelles méthodes n'ont pas vocation de remplacer les professionnels du domaine mais plutôt de les épauler avec des données qui deviennent de plus en plus large et à garder les prêts accessible à ceux qui le méritent le plus. Ces algorithmes sont capables de repérer des tendances dans les données, des corrélations entres différents paramètres et les comportements et decision des clients, créer un modèle statisque de ce qui représente un bon prêt et en fin de compte prédire les chances qu'un client a de faire défaut sur un prêt donné. Les machines ont un clair avantages sur nous en terme de puissance et de vitesse de calcul, les nouveaux outils apportés par l'apprentissage automatique en particulier et la science des données en général se sont donc vite imposés dans le flux de travail des banques et autres institutions financières avec un potentiel énorme pour rendre les prêts plus accessibles et plus sûrs.

Imaginez devoir analyser manuellement des centaines voir des milliers de demande en fonction d'une quantité colossale de variables, en plus des experts en finances et en économie qu'on suppose qu'ils ont déjà, une banque aura besoin de statisticiens, probablement de juristes et de gens qui connaissent la réalité des différents domaines proposé par les démandeurs, des \textit{"experts de domaine"}. Par exemple, il y'a encore 30 ans beaucoup ne donnait pas cher de la peau de l'internet et des nouvelles compagnie qui se basaient sur lui. Quand on voit ce qu'est devenu Facebook aujourd'hui, très mauvais pari effectivement. De manière similaire, même des experts de la finance se moquaient du \textbf{Bitcoin} au début des années 2010 et ne cessaient de prédire sa chutte imminente. L'économie est une chose complexe où des tendances se créent et s'inversent à chaque décennie, prendre des décisions informées implique de se mettre constamment à jour avec ce monde.

La promesse de ces nouvelles techniques est de pouvoir comprendre et même prédire ces tendances de manière mathématiques. Dans ce mémoire, nous allons procéder à une étude approfondie de ces nouvelles techniques provenant de la science des données et de l'apprentissage automatique couvrant aussi bien leur fonctionnement que les algorithmes qui sont derrière, ainsi que leur application dans le monde réel. La dernière section de ce premier chapitre apporte plus de détails sur les spécifications actuelles du travail éffectué.
