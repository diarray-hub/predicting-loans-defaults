\chapter{Traitement des données}
\label{chapter3}
Avec l'intelligence artificielle en général et l'apprentissage automatique en particulier, les données sont toujours le nerf de la guerre, même plus important que les codes et les algorithmes les données d'entraînement détermineront à 80\% le comportement du système. C'est donc tout naturellement la partie où l'on consacre le plus de temps et d'énergie. Avoir des données en quantité ne suffit pas, il s'agit de trouver les meilleurs prédicateurs, la meilleure représentation, la meilleure normalisation\footnote{La normalisation refère au processus de transformation visant à reduire une colonne ou une table entière à une échelle jugée normale, par exemple faire un sorte que les valeur d'une colonne ne varie que entre zéro et un afin de les interprêter comme des probabilités.} etc...

Dans ce chapitre, j'explique les différents algorithmes et techniques de traitement des données que j'ai appliqué à l'ensemble de données original formé par Home Credit Group. Notez que ce chapitre ne couvre que les techniques que j'ai utilisé pour ce projet spécifique, il ne s'agit pas d'une introduction au traitement des données donc il sera plus concis et détaillé.

\section{Contexte}
\label{chap1.section1}
Dans notre société contemporaine, l'économie à l'échelle mondiale se répose de plus en plus sur les prêts. Cette situation est notamment favorisée par le fait qu'aujourd'hui toutes les économies se basent sur une forme de capitalisme\footnote{Le capitalisme est un système économique caractérisé par la propriété privée des moyens de production, des marchés concurrentiels et la recherche du profit.}, les économies socialistes\footnote{Le socialisme est un système politique et économique dans lequel la propriété et les moyens de production sont détenus en commun, généralement contrôlés par l'État ou le gouvernement. Le socialisme repose sur l’idée que la propriété commune ou publique des ressources et des moyens de production conduit à une société plus égalitaire.} étant souvent taxée comme restrictives de liberté et empêchant le dévéloppement économique des individus au profit de celui de l'état. Cette mentalité inspirée surtout par \textbf{\textit{"The american dream"}}, le rêve américain en français, a pas mal chambouler les normes, maintenant on a tous un peu l'idée que pour être quelqu'un il faut avoir une villa, une voiture et des bouts de terre à son nom.

Ce changement de mentalité a été le point de départ d'importants changements dans l'économie mondiale. Le rêve américain s'est démocratisé et est devenu un rêve mondial, apportant l'idée que chaque personne a la liberté et la possibilité de réussir et d’accéder à une vie meilleure. C'est donc à partir de là que le système de prêt est devenu réellement LE pilier de l'économie mondiale avec un impact absolument démésuré. D'après la banque mondiale, le crédit mondial au secteur privé représentait environ 97,6\% du PIB mondial en 2020 (\cite{worldbank2024domestic}).

Les prêts ont definitivement changé de statut et l'économie mondiale de visage, les individus emprûntent de l'argent pour s'acheter une maison, une boutique, une voiture, en bref vivre le rêve américain mais aussi pour faire des études supérieures. L'americain moyen passe typiquement les premières années de sa vie professionelle à rembourser son prêt étudiant. Au delà des individus les gouvernements contribuent à renforcer ce système souvent à travers des rélations complexes avec les banques.

Mais là où l'impact est le plus grand est sans aucun doute le secteur de l'entreprenariat privé. Il est vraiment impressionant de se dire qu'il y a des entreprises aujourd'hui évaluées à des centaines de millions voir des milliards de dollars qui partent vraiment juste d'une bonne idée et d'un investissement. Pendant la crise du coronavirus, les crédits ont été le bienfaiteur qui a sauvé des milliers de petites et moyennes entreprises à travers le monde et paradoxalement ils ont aussi été la raison qui a précipité la chutte de centaines d'entre elles.

Les dettes, c'est souvent le commencement de la ruine. Après la crise du covid-19, le FC Barcelone (incontestablement un des plus grands clubs de football de l'histoire) annoncait une dette globale de plus de 2 milliards d'euros, ce fut le début d'une période très difficile que le club traverse encore aujourd'hui. Pas seulement dans le domaine du sport, tellement d'entreprises et organisation ont dû déposer le bilan à cause de leur dettes, les prêts font les grands et les dettes les détruisent. On ne crée pas de richesse à partir de rien, ce qui fait que même les banques et les institutions prêteuse sont en réalité souvent très endettées. Pour rappel, la banque d'investissement des frères Lehman a déposé son bilan en raison de son exposition massive aux dettes hypothécaires à risque, ce qui a conduit à la grande crise financière de 2008.

C'est probablement après cet évènement que les gens et en particulier les américains ont réellement réalisé que l'énorme croissance économique que l'on a connu après la seconde guerre mondiale s'est faite sur des dettes cumulées énormes des grandes institutions et qu'on ferait mieux ne pas jouer avec ça. Suivant la crise plusieurs lois ont été votées ou modifiées un peu partout dans le monde afin d'eviter un tel désastre dans le futur, mettant l'accent sur la protection du client mais aussi la transparence des procedures.

Vous pensez peut-être que toutes ces choses sont des problèmes de la société contemporaine mais détrompez-vous, j'ai été surpris d'apprendre que bien qu'ils n'avaient jamais atteint un tel niveau d'importance auparavant différents systèmes de prêt ont en réalité soutenu les économies de plusieurs civilisations par le passé. Bien que ces concepts et lois ont évolué avec le temps, ils sont enfaîte extrêmement anciens. Autour de moins 2000 ans avant Jésus Christ en Mésopotamie, les temples et les palais prêtaient du grain aux agriculteurs et aux commerçants avec des formes de contrat incluant des taux d'interêt, des clauses et des délais, un peu comme les prêts d'aujourd'hui. Le \textbf{code d'Hammurabi}, un texte légal Babylonien composé entre 1755 et 1750 avant Jésus Christ incluait déjà des règles sur les prêts et les taux d'intérêt. Il s’agit du texte juridique le plus long, le mieux organisé et le mieux conservé de l’ancienne Mésopotamie. Il est écrit dans le vieux dialecte babylonien de l'akkadien, prétendument par Hammurabi, sixième roi de la première dynastie de Babylone.

Ça fait beaucoup de choses qu'un informaticien de la génération Z n'est pas vraiment censé savoir mais s'il y'a une chose que j'ai bien compris c'est qu'avec le model économique actuelle le système de prêt est à proteger absolument au risque de l'éffondrement de l'économie mondiale. Dans cette optique, une décision en particulier devient plus crucial que le reste: À qui accorder un prêt? Pour proteger à la fois le client et le prêteur, il est primordial de pouvoir déterminer avec précision quel client est ou sera en capacité de rembourser tel prêt. Voilà qui nous amène à notre thème maintenant, dans la prochaine section nous allons discuter le comment maintenant.
\section{Approches}
\label{chap1.section2}
Les banques sont tenues de respecter les normes réglementaires qui imposent des pratiques de prêt prudentes. Pouvoir prédire de manière précise qui peut ou ne peut pas rembourser tel prêt aident à respecter ces normes et à éviter les pénalités.

Pour prédire les défauts de paiement, les banques analysaient traditionnellement  divers facteurs, parmi lesquels:

\begin{itemize}
    \item Historique de crédit: un enregistrement détaillé du comportement d’emprunt et de remboursement passé du demandeur.
    \item Revenu et emploi: un revenu constant et suffisant indique une capacité à rembourser le prêt.
    \item Ratio dette/revenu: il mesure le fardeau de la dette du demandeur par rapport à son revenu.
    \item Valeur de la garantie: pour les prêts garantis, la valeur de la garantie est cruciale en cas de défaut de paiement de l'emprunteur.
\end{itemize}

Typiquement une demande de prêt est analysée méticuleusement suivant ces différents aspects par des professionels du domaine qui en fonction de ces éléments decident d'accorder ou de refuser le prêt au demandeur. Vous l'aurez compris, les antécédents priment beaucoup dans la procedure, un jeune diplômé qui vient d'obtenir son premier contrat à durée indéterminée (CDI) n'a presque aucune chance de pouvoir obtenir un prêt, à moins d'avoir une garantie de fer.

Des méthodes un peu plus modernes peuvent également inclure des données comportementales, telles que les habitudes de dépenses et même l'activité sur les réseaux sociaux. Oui ça commence à faire un peu beaucoup, ce qui nous amène encore une fois directement au thème de ce mémoire. L'étude d'une demande de prêt est une procédure délicate qui augmente en complexité avec différents facteurs. Les données à analyser aussi croissent en complexité et surtout en quantité multipliant le risque d'errreur humaine dans la procédure.

Des mauvaise prévisions repétées peuvent avoir de graves répercussions:

\begin{itemize}
    \item Augmentation des pertes sur prêts: un nombre plus élevé de défauts de paiement a un impact direct sur les résultats financiers de la banque.
    \item Taux d’intérêt plus élevés : pour couvrir les pertes potentielles, les banques pourraient augmenter les taux d’intérêt pour tous les emprunteurs, rendant les prêts plus chers.
    \item Resserrement du crédit : les banques pourraient devenir plus conservatrices dans leurs prêts, ce qui rendrait plus difficile l’obtention de prêts pour les particuliers et les entreprises.
    \item Dommages à la réputation : des taux de défaut élevés peuvent nuire à la réputation d’une banque, affectant la confiance des clients et ses activités futures.
\end{itemize}

\textbf{L'apprentissage automatique} ou apprentissage machine, un sous-domaine de l'intelligence artificiel centré sur la création d'algorithme qui puisse extraire, interprêter et généraliser des connaissances à partir des données, a apporté de nouvelles méthodes visant à assister cette procédure par les machines, améliorant considérablement la précision et l’efficacité. Ces nouvelles méthodes n'ont pas vocation de remplacer les professionnels du domaine mais plutôt de les épauler avec des données qui deviennent de plus en plus large et à garder les prêts accessible à ceux qui le méritent le plus. Ces algorithmes sont capables de repérer des tendances dans les données, des corrélations entres différents paramètres et les comportements et decision des clients, créer un modèle statisque de ce qui représente un bon prêt et en fin de compte prédire les chances qu'un client a de faire défaut sur un prêt donné. Les machines ont un clair avantages sur nous en terme de puissance et de vitesse de calcul, les nouveaux outils apportés par l'apprentissage automatique en particulier et la science des données en général se sont donc vite imposés dans le flux de travail des banques et autres institutions financières avec un potentiel énorme pour rendre les prêts plus accessibles et plus sûrs.

Imaginez devoir analyser manuellement des centaines voir des milliers de demande en fonction d'une quantité colossale de variables, en plus des experts en finances et en économie qu'on suppose qu'ils ont déjà, une banque aura besoin de statisticiens, probablement de juristes et de gens qui connaissent la réalité des différents domaines proposé par les démandeurs, des \textit{"experts de domaine"}. Par exemple, il y'a encore 30 ans beaucoup ne donnait pas cher de la peau de l'internet et des nouvelles compagnie qui se basaient sur lui. Quand on voit ce qu'est devenu Facebook aujourd'hui, très mauvais pari effectivement. De manière similaire, même des experts de la finance se moquaient du \textbf{Bitcoin} au début des années 2010 et ne cessaient de prédire sa chutte imminente. L'économie est une chose complexe où des tendances se créent et s'inversent à chaque décennie, prendre des décisions informées implique de se mettre constamment à jour avec ce monde.

La promesse de ces nouvelles techniques est de pouvoir comprendre et même prédire ces tendances de manière mathématiques. Dans ce mémoire, nous allons procéder à une étude approfondie de ces nouvelles techniques provenant de la science des données et de l'apprentissage automatique couvrant aussi bien leur fonctionnement que les algorithmes qui sont derrière, ainsi que leur application dans le monde réel. La dernière section de ce premier chapitre apporte plus de détails sur les spécifications actuelles du travail éffectué.

\section{Introduction formelle du thème}
\label{chap1.section3}
Bien que ces nouvelles approches ne sont pas encore largement répendues et utilisées par les banques au Mali, elles sont beaucoup plus communes en occident et déjà adoptées par les grandes institutions. Donc des progrès déjà significatifs ont été réalisés dans le domaine, enfaîte la prediction des défauts de paiement est l'une des plus vielles application de l'apprentissage automatique, un classique.

Ce que je propose avec ce thème est donc une analyse de cet état de l'art et du flux de travail associé, pour celà j'ai implémenté certaines des dernières techniques de prétraitement et d'analyse des données ainsi que les modèles d'apprentissage automatiques qui ont atteint l'état de l'art dans cette tâche. J'ai voulu diverger de la majorité des chercheurs en testant aussi des algorithmes moins conventionnels et moins populaires pour la prediction de défaut de paiement, qui m'ont, au final, agréablement surpris par leur résultats au moment des tests.

Au total cinq (5) modèles d'apprentissage automatique différents ont été entraînés et testés, des valeurs sûres comme la régression logistique, la forêt aléatoire et le boost de gradient mais aussi les réseaux de neuronnes artificiels (souvent mis de côté pour ce genre de tâche en raison de leur complexité) et la détection d'anomalie qui à la base n'est pas un algorithme de classification mais qui à démontré la meilleure précision avec des données déséquilibrées. Le chapitre 4 va en profondeur dans les processus d'entraînement et les différences conceptuelles de chacun de ces modèles ainsi que les implémentations et stratégies que j'ai utilisé.

Ces cinq modèles ont tous été entraînés et testés suivant le même canevas ou \textit{pipeline} et avec le même ensemble de données, afin que les performances ne puisse différer que de part les caractéristiques intrinsèques de chaque modèle limitant leur performances après un certains nombre d'étapes d'optimisation ainsi évitant de biaser leur évaluation dans un système ou un modèle est avantagé par rapport aux autres.

L'ensemble de données en question est constituée d'exactement 100 variables indépendantes ou prédicateurs (\textbf{features}) et d'une variable dépendante (celle qu'on cherche à prédire, \textbf{target}), à savoir si oui ou non le client en question a remboursé son prêt. Totalisant 1.526.659 exemples étiquétés qui seront divisés en deux groupes pour l'entraînement et le test. Ces données ont été obtenu après une étape d'ingénierie de prédicateurs ou création de prédicateurs et une étape de sélection et de prétraitement des variables indépendantes (détaillées dans le chapitre 3) que j'ai appliqué aux données brutes fournies par Home Credit Group, une institution financière non bancaire internationale fondée en 1997 en République tchèque et dont le siège est aux Pays-Bas, lors d'une competition qu'ils organisaient sur la plateforme mondiale des scientifiques des données et d'ingénieurs en apprentissage automatique, \textbf{Kaggle} (\cite{herman2024home}). La société opère dans 9 pays et se concentre sur les prêts avec une attention particulière aux personnes ayant peu ou pas d'antécédents de crédit.

Ceci couvre en grosso modo ce qu'il faut savoir sur le travail éffectué. Dans le deuxième chapitre nous nous intéressons à la définition et à la compréhension des concepts phares de la science des données qui sont mentionnés ou utilisés dans le reste du document. Ce chapitre vise à familiariser le lecteur avec ces notions avant de rentrer dans la partie technique du document, vous pouvez le voir comme une sorte de glossaire mais un peu plus détaillé.

\clearpage