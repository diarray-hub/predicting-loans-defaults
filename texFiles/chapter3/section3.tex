\section{Prétraitement des données}
\label{chap3.section3}
C'est la dernière étape du traitement des données, à cette étape du pipeline nous cherchons une représentation finale pour les données un format directement utilisable lors de l'entraînement, des tests et du déploiement. Traditionnellemnt c'est l'étape qui précède directement l'entraînement des modèles. Généralement, elle consiste en trois étapes:

\begin{enumerate}
    \item \textbf{Gérer les valeurs manquantes}: Lors de la selection des prédicateurs on a éliminé les colonnes contenant plus de 70\% de valeurs manquantes. Des colonnes pourrait être en dessous de ce seuil mais contenir un nombre assez important de valeurs manquante. Pour ne pas avoir à supprimer les examples à qui appartiennent les valeurs manquantes, il est nécéssaire de trouver une réprésentation pour ces valeurs manquantes. Ma priorité était de garder tous les exemples donc les valeurs manquantes on été remplacées par deux valeurs volontairement hors de la distribution (\textbf{-9999} pour les valeurs colonnes de type numérique et le token "<N/A>" pour les prédicateurs non numériques.
    \item \textbf{Encodage}: Comme défini dans la section \ref{chap2.section2}, l'encodage est une étape où l'on cherche une représentation numérique pour toutes les variables qui ne le sont pas par nature. Au vue de la grande dimensionnalité des données que l'on possède, j'ai opté pour la forme la plus simple d'encodage possible en attribuant un identifiant numérique unique (dans sa colonne) à chaque variable de type catégorique, jour de la semaine, mois, saison ou chaîne de caractère. Mais les valeurs sont déterminées par l'ordre d'occurence des différentes valeurs dans le sous-ensemble de référence. Par exemple, si la première saison à apparaître dans le sous ensemble de référence est l'automne alors l'automne sera représenter comme 0 et ainsi de suite.
    \item \textbf{Normalisation}: La normalisation standard consiste à transformer les colonnes de manière à ce que leur valeurs décrivent une distribution de moyenne 0 et d'écart type 1. Pour cela on peut calculer la moyenne et la variance de chaque colonne et on remplace chaque valeur grâce à la formule \[x = \frac{(x - \bar{x})}{\sqrt{Variance}} = \frac{(x - \bar{x})}{\sigma}\]
\end{enumerate}

Enfin il est important de noter que les données sont aléatoirement divisées en deux groupes distincts avant l'entraînement, un ensemble d'entraînement et un ensemble de test. L'ensemble de test, prévu pour évaluer le modèle après entraînement correspond à 5\% des 1.526.659 demandes. 