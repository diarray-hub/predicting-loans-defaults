\section{Les données : Oxygène du nouveau monde}
\label{chap2.section1}
Depuis l’avènement du World Wide Web permettant aux individus du monde entier d'établir des rélations et de partager des connaissances, le phénomène du \textbf{big data}\footnote{Le terme "big data" fait référence à des ensembles de données si volumineux et complexes qu'ils dépassent les capacités des outils traditionnels de gestion et d’analyse. Ce phénomène a pris de l'ampleur avec l'explosion d'Internet et des réseaux sociaux. Chaque jour, des quintillions d'octets de données sont générés à partir de diverses sources: réseaux sociaux, sites web, capteurs, transactions financières, et plus encore.} a transformé notre monde. Chaque clic, chaque interaction, chaque transaction génère des données qui, une fois collectées et analysées, offrent une mine d’informations précieuses. Le domaine de la science des données a émergé pour exploiter ce potentiel.

La science des données et l'intelligence artificiel\footnote{L'intelligence artificiel est la science qui visent à créer des machines capables d'éfffectuer des tâches qui requièrent normalement de l'intelligence pour être accomplies ou capables d'agir d'une manière qui serait considérée intelligente si un humain agissait de la sorte} ont permit de créer certains des outils le plus impactant du 21 ème siècle grâce à ces données. 

Prenons par exemple une autre application classique de la science des données: les systèmes de recommendations. Les système de recommendations sont des modèles d'apprentissage automatique capable d'analyser et de suivre les préférences et intérêts des gens et de leur proposer des choses similaire (qu'ils sont donc susceptibles d'apprécier). Ces systèmes sont utilisés pour faire de la publicité ciblée ou par les réseaux sociaux pour proposer du contenu aux internautes. Ces systèmes ont atteint un tel niveau que beaucoup de gens sont convaincus d'être espionné. Si vous voulez connaître quelqu'un ou du moins savoir ce qui l'interêsse, prenez sont téléphone et ouvrez Youtube, facebook ou Tiktok pour les plus jeunes.

Parfois c'est éffrayant de réaliser tout ce que les grandes compagnies de la tech comme Google ou Facebook savent sur nous mais pourtant nous avons accepter les conditions d'utilisations et les règles de confidentialité. Vous connaissez surêment l'expression: \textit{"Internet n'oublie jamais"}, on ne peut plus être invisible aujourd'hui tant que l'on génère des données, chaque clic, chaque recherche, chaque contenu fournit des informations pouvant être utilisé pour savoir qui vous êtes, où vous êtes, ce que vous aimez ou pas et même ce que vous faites en ce moment même. C'est ça le nouveau monde, c'est dans ça que l'on vit et c'est ça le pouvoir et le science des données.

Après cette introduction, la prochaine section cherche à définir les différentes notions liées à cette science pour pouvoir transitionner de manière fluide vers la partie technique du mémoire. 