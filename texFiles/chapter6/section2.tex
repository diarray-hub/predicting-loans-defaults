\section{Discussion}
\label{chapter6.section2}
Dans cette dernière section je veux surtout mettre en lumières les principaux chanllenges rélatifs à mes recherches et conclure brièvement avec les futures perspectives de la recherches et quelques remarques personnelles.

Je pense qu'en y repensant, les difficultés majeures que j'ai rencontré dans mes recherches si je ne devais en citer que deux, ce serait dans l'ordre:

\begin{enumerate}
    \item \textbf{Le traitement des données}: C'était bien la toute première fois que j'ai eu à travailler avec un ensemble de données si enorme et si complexe. Utiliser la création automatique de prédicateur était, à la base, un choix que j'ai fais pour compenser mon manque d'expérience dans cette tâche et cela s'est aussi avéver être un calvaire à gérer et surtout à optimiser. Avec un ensemble de données si grand, une machine ordinaire était juste incapable d'exécuter les différents algorithmes qui étaient liés au traitement des données et même les machines virtuelles offertes par Kaggle avaient atteint leur limites. C'était un vrai casse-tête de trouver et d'implémenter des stratégies pour essayer de gérer au mieux mes ressources. Résultat c'est la tâche qui s'est avérer la plus complexe finalement.
    \item \textbf{Gérer le déséquilibre}: Pas que ça a été compliqué puisque je savais déjà à quoi m'attendre et comment y remédier, plutôt que gérer le déséquilibre des données m'a forcé à faire des choix au niveau des modèles et des métriques et surtout à faire des compromis entre les différent métriques. Ce qui au final a prolongé les temps d'entraînement et d'évaluation des modèles.
\end{enumerate}

À titre personnel, l'idée d'utiliser un modèle de détection d'anomalies pour tacler les problèmes de classification bianire déséquilibré était une agréable considération dans mes recherches, je pense que bien qu'elle soit beaucoup moins éfficace en termes de temps et de ressources, c'est une méthode qui mériterait d'être plus étudiée et plus citée comme une solution pour gérer les ensembles de données déséquilibré. Je tiens toutefois à préciser que les résultats présentés dans ce document ne sont pas une limite dure des performances des différents modèles et algorithmes et qu'ils ne doivent en aucun cas être interprêté comme \textbf{l'état de l'art} de l'apprentissage automatique dans la tâche de prédiction des défauts de paiement sur les prêts.

L'application des modèles d'apprentissage automatique à la prédiction des défauts de paiement est un exemple entre autres des aspects de la technologie dans lesquels nous (en tant qu'africain et malien) sommes très en retard par rapport aux occidentaux et asiatiques. Ce qui est vraiment dommage quand on voit le potentiel de transformations de l'intelligence artficielle et de ses sous-domaines aujourd'hui et la vitesse à laquelle ils évoluent. Je pense que dans un futur proche et même plutôt très proches nous devons mettre en place des solutions pour former les gens sur comment créer et utiliser ces systèmes là au risque d'accumuler encore plus de retard dans un monde qui définitivement ne nous attendra pas.