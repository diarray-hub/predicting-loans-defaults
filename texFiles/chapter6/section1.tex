\section{Résumé}
\label{chapter6.section1}
À travers ce mémoire, j'ai essayé d'étudier et de diagnostiquer le système de prêt moderne, ses dessous, ses origines et surtout son impact sur l'économie mondiale. Le système de prêt est aujourd'hui le support principal de l'économie moderne servant à créer des startups, financer des études, acheter des maisons etc... Il est actuellement très difficile de citer un seul domaine d'accroîssement économique qui n'en est pas presque dépendant. 

Autant la tâche est plutôt simple quand le demandeur du prêt est une grande organisation, autant elle dévient délicate quand il s'agit d'individus cherchant à financer leurs projets. Accorder ou refuser un prêt à un individus est une tâche qui se repose sur une enquête traditionnellement éffectuée par les institutions prêteuses sur les démandeurs, dans ce processus l'absence d'antécédents est souvent très péjoratif pour le demandeur. Pour protéger les instititions prêteuses et les demandeurs de prêt il est nécéssaire de pouvoir attester la capacité d'un demandeur à rembourser le prêt qu'il demande.

Home Credit Group, une institution financière non bancaire internationale, a fait de ces demandeurs avec peu ou pas d'antécédents sa priorité. En février 2024, ils ont lancé une compétition sur la plateforme Kaggle où ils ont rendu publique les données de leurs enquêtes sur plus d'un million et demi de demandeurs de divers zones géographiques, parcours et antécédents qui s'étaient vu accorder un prêt, ainsi que le statut final de leurs prêts (remboursé/non remboursé) avec la tâche de créer un modèle d'apprentissage pour distinguer les bons des mauvais prêts. Ce ensemble de données qu'ils ont publié a été le point de départ de mes recherches.

Le but des recherches était ensuite de fournir une exploration approfondie des toutes dernières méthodes d'apprentissage automatique qui ont été récemment utilisées ou qui sont actuellement étudiées pour la tâche de prédiction des défauts de paiement; en utilisant les données fournies par Home Credit Group. En expliquant le plus clairement possible les algorithmes derrière ces différentes méthodes, les techniques et concepts associés ainsi que les implémentations que j'ai fait ou utilisés.

Pour cela j'ai implémenté, entrainé et testé 5 modèles d'apprentissage automatique dont quatre modèles de classification et un modèle de détection d'anomalies ainsi que plusieurs autre techniques de traitement des données. Les modèles entraînés ont ensuite été évalués avec des métriques comme la justesse de classification, la précision, le rappel et l'AUC.